%%!TEX root = ./UserManual.tex
  
\chapter{Parallelization}

To speed up simulations, Stride tries to distribute certain workloads across multiple processors or processor cores. These workloads include transmitting diseases and updating people's presence in clusters and are typically ``embarrassingly parallel'' problems.

The task of slicing the workload up into chunks and assigning appropriate machine resources to those chunks is delegated to external libraries in varying degrees, depending on the library used.

At the time of writing, three parallelization libraries are supported by Stride: OpenMP, Intel Threading Building Blocks (TBB), and the standard library's threading module. Additionally, a serial implementation of the parallelization libraries' functionality is offered.

\section{Configuring Stride's parallelization library}

There are two ways to configure Stride's choice of parallelization library:

\begin{enumerate}
	\item The \texttt{STRIDE\_PARALLELIZATION\_LIBRARY} environment variable can be set to one of \texttt{OpenMP}, \texttt{TBB}, \texttt{STL} or \texttt{none} at configure-time. This explicitly instructs the build process to use the specified parallelization library.
	
	\item If \texttt{STRIDE\_PARALLELIZATION\_LIBRARY} is not set, then the build system will automatically pick an appropriate parallelization library depending on which libraries you have installed.
\end{enumerate}

The following subsections provide some more detail on how to set up Stride for specific parallelization libraries.

\subsection{OpenMP}

OpenMP relies on both compiler support and a runtime library. If your machine has both, then OpenMP is used by default. Alternatively, you can explicitly use OpenMP by setting the \texttt{STRIDE\_PARALLELIZATION\_LIBRARY} environment variable to \texttt{OpenMP} at configure-time.

\subsection{Intel Threading Building Blocks}

To use Intel Threading Building Blocks (TBB), you'll need to install the TBB libraries on your machine. You specify TBB explicitly by setting environment variable \texttt{STRIDE\_PARALLELIZATION\_LIBRARY} to \texttt{TBB} at configure-time. If TBB is installed and OpenMP is not, and if \texttt{STRIDE\_PARALLELIZATION\_LIBRARY} is not set, then TBB is used as the default parallelization library.

\subsection{Standard library threads}

If neither OpenMP nor TBB are installed, then Stride will use a custom parallelization library based on the C++ standard library's threading module. To use this implementation even if OpenMP and/or TBB are installed, set the value of the \texttt{STRIDE\_PARALLELIZATION\_LIBRARY} environment variable to \texttt{STL}.

\subsection{No parallelization}

To use a serial implementation of all parallelization algorithms that rely on the presence of one of the parallelization libraries listed above, set environment variable \texttt{STRIDE\_PARALLELIZATION\_LIBRARY} to \texttt{none}.

Note that the build system will \emph{never} use this option by default, as the implementation based on standard library threads does not have any dependencies other than Stride's required dependencies.

\section{Performance comparison}

OpenMP, TBB and our custom implementation based on standard library threads are more or less equivalent in terms of performance. Using no parallelization library slows the simulator down, but not dramatically so.

Figure \ref{fig:parallelization-mean-runtimes} shows the mean run-times of the \texttt{run\_popgen\_small.xml}, \texttt{run\_popgen\_medium.xml} and \texttt{run\_popgen\_large.xml} configurations for each parallelization library. These configurations generate a population and then run the simulator on the generated population for one hundred days. They are identical in every way except for the size of the population that is generated.

Each configuration was run ten times per parallelization library. Mean run-times are represented in figure \ref{fig:parallelization-mean-runtimes}. The relative standard deviation for each cell in the table is less than four percent. The measurements were produced by an Intel Core i7-6700K CPU running Ubuntu 17.04, gcc 6.3.0 and TBB 4.4.

\begin{figure}[h]
	\begin{tabular}{r|r|r|r|r}
		\textbf{Population size} & \textbf{OpenMP} & \textbf{TBB} & \textbf{STL} & \textbf{none} \\
		100,000 & 6.00s & 6.09s & 5.80s & 6.69s \\
		500,000 & 53.01s & 53.34s & 52.08s & 57.97s \\
		1,000,000 & 169.69s & 166.81s & 167.91s & 178.97s
	\end{tabular}
	\label{fig:parallelization-mean-runtimes}
	\caption{Mean run-times for each parallelization library.}
\end{figure}

What stands out from this table is that our standard library threads--based implementation is competitive with OpenMP and TBB performance-wise, despite using a na\"ive static partitioning scheme.

This can likely be attributed to the homogeneity of the workloads that are parallelized: at the time of writing, parallelization functions are used in such a way that arbitrarily chosen but equally-sized chunks of items to process take roughly the same amount of time to process.

OpenMP and TBB have much more intelligent schedulers than our simple standard library threads--based implementation, but it's hard for them to outperform our simple static scheduler because its behavior is optimal for homogeneous workloads, which the workloads offered by Stride approximate well.