%%!TEX root = ./UserManual.tex
  
\chapter{Parallelization}

To speed up simulations, Stride tries to distribute certain workloads across multiple processors or processor cores. These workloads include transmitting diseases and updating people's presence in clusters and are typically ``embarrassingly parallel'' problems.

The task of slicing the workload up into chunks and assigning appropriate machine resources to those chunks is delegated to external libraries in varying degrees, depending on the library used.

At the time of writing, three parallelization libraries are supported by Stride: OpenMP, Intel Threading Building Blocks (TBB), and the standard library's threading module. Additionally, a serial implementation of the parallelization libraries' functionality is offered. These options are elaborated on by the sections below.

\section{OpenMP}

\section{Intel Threading Building Blocks}

\section{Standard library threads}

\section{No parallelization}

\section{Performance comparison}