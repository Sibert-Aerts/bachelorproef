%%!TEX root = ./UserManual.tex
  
\chapter{Parallelization}

To speed up simulations, Stride tries to distribute certain workloads across multiple processors or processor cores. These workloads include transmitting diseases and updating people's presence in clusters and are typically ``embarrassingly parallel'' problems.

The task of slicing the workload up into chunks and assigning appropriate machine resources to those chunks is delegated to external libraries in varying degrees, depending on the library used.

At the time of writing, three parallelization libraries are supported by Stride: OpenMP, Intel Threading Building Blocks (TBB), and the standard library's threading module. Additionally, a serial implementation of the parallelization libraries' functionality is offered.

\section{Configuring Stride's parallelization library}

There are two ways to configure Stride's choice of parallelization library:

\begin{enumerate}
	\item The \texttt{STRIDE\_PARALLELIZATION\_LIBRARY} environment variable can be set to one of \texttt{OpenMP}, \texttt{TBB}, \texttt{STL} or \texttt{none} at configure-time. This explicitly instructs the build process to use the specified parallelization library.
	
	\item If \texttt{STRIDE\_PARALLELIZATION\_LIBRARY} is not set, then the build system will automatically pick an appropriate parallelization library depending on which libraries you have installed.
\end{enumerate}

The following subsections provide some more detail on how to set up Stride for specific parallelization libraries.

\subsection{OpenMP}

OpenMP relies on both compiler support and a runtime library. If your machine has both, then OpenMP is used by default. Alternatively, you can explicitly use OpenMP by setting the \texttt{STRIDE\_PARALLELIZATION\_LIBRARY} environment variable to \texttt{OpenMP} at configure-time.

\subsection{Intel Threading Building Blocks}

To use Intel Threading Building Blocks (TBB), you'll need to install the TBB libraries on your machine. You specify TBB explicitly by setting environment variable \texttt{STRIDE\_PARALLELIZATION\_LIBRARY} to \texttt{TBB} at configure-time. If TBB is installed and OpenMP is not, and if \texttt{STRIDE\_PARALLELIZATION\_LIBRARY} is not set, then TBB is used as the default parallelization library.

\subsection{Standard library threads}

If neither OpenMP nor TBB are installed, then Stride will use a custom parallelization library based on the C++ standard library's threading module. To use this implementation even if OpenMP and/or TBB are installed, set the value of the \texttt{STRIDE\_PARALLELIZATION\_LIBRARY} environment variable to \texttt{STL}.

\subsection{No parallelization}

To use a serial implementation of all parallelization algorithms that rely on the presence of one of the parallelization libraries listed above, set environment variable \texttt{STRIDE\_PARALLELIZATION\_LIBRARY} to \texttt{none}.

Note that the build system will \emph{never} use this option by default, as the implementation based on standard library threads does not have any additional dependencies compared to Stride's required dependencies.

\section{Performance comparison}

\section{A note on reproducibility}
